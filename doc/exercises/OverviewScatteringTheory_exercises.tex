\documentclass[12pt]{article}

% set up paper size

\setlength{\textwidth}{6.9in}
\setlength{\textheight}{9.5in}
\setlength{\evensidemargin}{-0.2in}
\setlength{\oddsidemargin}{-0.2in}
\setlength{\topmargin}{-0.8in}
\setlength{\parindent}{0pt}
\setlength{\itemsep}{0pt}
\def\pls{$^+$}
\def\ts{\textstyle\mathstrut}
\def\askip{\hspace*{1in}}
\renewcommand{\baselinestretch}{1.2}

\setlength{\tabcolsep}{10pt}
\newcommand{\ol}[1]{\overrightarrow{\bf #1}}
\newcommand{\I}{\item}
\newcommand{\be}{\begin{enumerate}}
\newcommand{\ee}{\end{enumerate}}
\newcommand{\bi}{\begin{itemize}}
\newcommand{\ei}{\end{itemize}}


\newcommand{\beq}[0]{\begin{small}\[}
\newcommand{\eeq}[0]{\]\end{small}}
\newcommand{\beqa}[0]{\begin{small}\begin{eqnarray*}}
\newcommand{\eeqa}[0]{\end{eqnarray*}\end{small}}
\newcommand{\bsm}[0]{\begin{small}}
\newcommand{\esm}[0]{\end{small}}
\newcommand{\bst}[0]{\begin{small}}
\newcommand{\est}[0]{\end{small}}
\newcommand{\bfn}[0]{\begin{footnotesize}}
\newcommand{\efn}[0]{\end{footnotesize}}

\def\nab{\overrightarrow{\nabla}}
\def\galnab{\stackrel{\leftrightarrow}{\nabla}}



\newcommand{\rvec}{{\bf r}}
\newcommand{\xvec}{{\bf r}}
\newcommand{\pvec}{{\bf p}}
\newcommand{\qvec}{{\bf q}}
\newcommand{\kvec}{{\bf k}}

\newcommand{\adag}{a^\dagger}
\newcommand{\ad}{a^{\dagger}}

\newcommand{\Nmax}{N_{\rm max}}


\newcommand{\la}{\langle}
\newcommand{\ra}{\rangle}

\newcommand{\Vext}{v_{\rm ext}}
\newcommand{\Egs}{E_{\rm gs}}
\newcommand{\fmi}{\mbox{\,fm}^{-1}}

\newcommand{\Hhat}{{\widehat H}}
\newcommand{\Nhat}{{\widehat N}}
\newcommand{\Vhat}{{\widehat V}}
\newcommand{\That}{{\widehat T}}
\newcommand{\Uhat}{{\widehat U}}
\newcommand{\Ohat}{{\widehat O}}
\newcommand{\Shat}{{\widehat S}}
\newcommand{\rhat}{\widehat{\bf r}}
\newcommand{\jhat}{\widehat\jmath}
\newcommand{\khat}{\widehat k}
\newcommand{\hhat}{\widehat h}
\newcommand{\nhat}{\widehat n}

\newcommand{\kf}{k_{\scriptscriptstyle\rm F}}



\newcommand{\flow}{s}


%\pagestyle{empty}    % no page numbers

\usepackage{graphicx}  % has \includegraphics, etc.
\usepackage{bm}
\usepackage{amssymb}
\usepackage[long,us,hhmmss]{datetime}

%###########################################################################%
%###########################################################################%

% Now for the manuscript.  The text always starts with a \begin{document}
%  line and ends with a \end{document} line.  Everything after \end{document}
%  is ignored.


\begin{document}

\raggedright

\begin{center}
{\large\bf
  PHY 982: Nuclear Forces \\
  Exercises for Module 1:\\
Overview of QCD and Scattering Theory} \\
   \relax [Last revised on \today\ at \currenttime.]
\end{center}

\vspace{.1in}
\textbf{Overview of exercises and discussion questions.}

The problems here range from basic and short (i.e., taking on the order of a few minutes to answer with little or no math) to more sophisticated and/or lengthy (marked with an asterisk). Many of these were shamelessly pilfered from the 2013 Nuclear TALENT course on Nuclear Forces taught by Dick Furnstahl and Achim Schwenk at the Institute for Nuclear Theory in Seattle. Comments on the pedagogy and logistics:
\bi
  \I Our underlying philosophy is that students learn most effectively when
  they actively fill in details of arguments and explicitly address conceptual
  questions with their classmates.  Some of the problems here are designed to lead the student
  to go back over particular lecture material to make sure it is understood,
  while others extend the lectures and still others introduce new topics or topics that were only lightly touched upon. (Note: besides discussing with your classmates, we strongly encourage google-sleuthing if you get stuck!)
  
  \I Given our time constraints, we do not attempt to develop the type of
  problem-solving skills that require students to struggle for hours over a single problem.  Rather,
  we try to point the way rather explicitly and let the student fill in details. 
  
  
  \I It is essential to try the exercises and to ask questions incessantly.
  Not everyone will be prepared to do all of the exercises completely, but with help from your instructors and fellow classmates, we are hopeful that everyone can take away the essential points.
  If you are unsure of what a word or phrase means in some context  or what a symbol stands for, please ask during the lectures or any of the instructors afterwards!
  

   \I
  The exercises are divided into categories (sometimes implicitly) 
  according to the type of problem: conceptual discussion questions, two-minute
  questions (if the material was understood, an answer is possible in
  a couple of minutes), basic skills problems, synthetic problems
  (putting skills together),
  rich context (real-life problems), and advanced problems (for those who already have
  additional background or problems that might take a long time). For all of these, we strongly encourage (demand!) that you work together. 

  \I
  The exercises will not be graded, apart from checking that you made a good-faith effort to work on them. We have decided not to provide written solutions, as our goal is to encourage you all to work together and reach a consensus. At the end of the semester (time permitting), we might revisit any open questions 


\ei


%###########################################################################%
%###########################################################################%
\newpage

\bigskip{\large\textbf{Questions on the Overview of QCD }}
\be
  \I Short exercises and discussion questions on QCD.
    \be
      \I With respect to what scale(s) are the $c,b,t$ quarks called heavy?

      \I Have you heard about the $s$ quark before? If yes, in what context?

      \I A possible way to ``see'' quarks and gluons is in jets. What
      happens in these events?

      \I Using the Particle Data Group website
      \texttt{http://pdg.lbl.gov/}, discuss which properties of the
      neutron and proton are similar and what are differences?
      What about for the three pions?

%      \I Explore some other hadrons using the Particle Data Group
%      website. Which fit into your picture of hadrons from the lecture
 %     and which don't?

      \I Which is more important in making a neutron more massive than
      a proton: the light quark mass difference or the electromagnetic
      contribution? Or do you think such considerations are too
      simplistic?

      \I What is the evidence for \emph{spontaneous} chiral symmetry 
      breaking in
        \be
          \I the mass spectrum of pseudoscalar ($J^\pi = 0^-$) mesons;
          \I the mass spectrum of vector and axial vector 
          ($J^\pi = 1^\mp$) mesons?
        \ee

      \I What is the evidence for \emph{explicit} chiral symmetry breaking
      in the  spectrum of pseudoscalar ($J^\pi = 0^-$) mesons? 
      
      \I If you and your friend each do a QCD calculation with the
      same diagrams but use $\alpha_s$ at different scales, will you
      get the same answer? If not, how could that happen?

      \I Does the running coupling in QCD mean that the QCD
      Hamiltonian is not unique? Would you say that if you used
      $\alpha_s$ at two different scales that you were using two
      different Hamiltonians?

      \I If the neutron lifetime is so short, why are there \emph{any}
      stable nuclei?

      \I One observes a marked resonance when a $\pi^+$ pion is
      scattering off a proton. Which baryon does this correspond to
      and at which energy of the $\pi^+$ does this occur (the proton
      is at rest)?

      \I At sufficient energy in proton-proton collisions it is
      possible to create a pion, $p + p \rightarrow p + n + \pi^+$. At
      which energy in the center-of-mass frame does pion production
      start?
    \ee
\ee
\newpage
\bigskip{\large\textbf{Questions on Scattering Theory}}
\be
  \I Short exercises on units and conversions and dimensional analysis.
   \be
      \I We typically use units in which $\hbar = c = 1$ and express
      quantities as powers of MeV or fm or both, using $\hbar c \approx
      197.33\,$MeV--fm to convert between them. If we take for the nucleon
      mass $M_N = 939\,$MeV/$c^2$, what is $\hbar^2/M_N$ numerically
      in terms of MeV and fm? 
      [Hint: This should be almost immediate if you insert the right factors
      of $c$.]

      \I For the scattering of equal mass (nonrelativistic) particles, 
      if the laboratory energy $E_{\rm lab}$ is related to the magnitude of the relative momentum 
      $k_{\rm rel}$ (i.e., the momentum each particle has in the center-of-mass
      frame) by $E_{\rm lab} = C k_{\rm rel}^2$, what is $C$?  If the mass
      is $M_N = 939\,$MeV, what is the value of $C$ in MeV--fm$^2$? 

      \I We write the partial-wave momentum space Schr\"odinger equation
      (following the conventions in Landau, \textit{Quantum Mechanics II})
      as
      \beq
        \frac{k^2}{2\mu}\la klm | \psi \ra + \frac{2}{\pi}\sum_{l'm'}
        \int_0^\infty\! dk'\,k'{}^2\, 
        \la klm | V | k'l'm' \ra \la k'l'm' | \psi \ra = E_k \la klm | \psi \ra \;,
      \eeq 
      what are the units of $V_{ll'}(k,k') \equiv   \la klm | V | k'l'm \ra$? In coordinate space the potential is local, $V(r)$, with units of MeV, and $k$ is given in $\fmi$.  If you see a plot in a journal
      article
      of $V_{ll'}(k,k')$ with units of fm, how would you convert it to the units you just found?
      [Hint: use part (a).]

      \I In Fig.~18 of the review by S.\ K.\ Bogner {\it et al.},
      Prog.  Nucl. Part. Phys. {\bf 65}, 94 (2010) the momentum-space
      matrix elements of different chiral effective field theory
      potentials are given in units of fm. Consider the value at zero
      relative momenta. For the EGM potentials this is given by
      $\tilde{C}_{^1S_0}$, see Eq.~(2.5) in EGM, Nucl. Phys.  {\bf
        A747}, 362 (2005). The values for $\tilde{C}_{^1S_0}$ are
      given in Table 2 of that paper in GeV$^{-2}$. How do you convert
      to fm units? Do the values for the matrix elements then match?

   \ee

  
  \I Short exercises reviewing basic scattering theory:
   \be
     \I What do ``on-shell'' and ``off-shell'' mean in the context of scattering?

     \I Under what conditions is a partial-wave expansion of the potential useful?

     \I Derive the standard result:
     \beq
       \frac{e^{i\delta_l(k)}\sin\delta_l(k)}{k}
            = \frac{1}{k\cot\delta_l(k) - i k}
     \eeq
     [Hint 1: First move 
  $e^{i\delta_l}$ to the denominator, then replace it by $\cos + i\sin$.]
  %\\ \relax
  %   [Hint 2: Divide top and bottom by $\sin$.]


     \I Given a potential that is not identically zero as $r\rightarrow\infty$ (e.g., a Yukawa),
     how would you know in practice where the asymptotic (large $r$) region starts?

    \I What is the physical interpretation of the relation between the (partial-wave)
    S-matrix and the scattering amplitude?  (Note that $S_l(k) = 1 + 2 i k f_l(k)$.) 

  %  \I Advanced: what is the structure of 
  %    the T-matrix near a bound-state energy?  
  %    [\emph{Or give the result
  %    and ask a question about it.}]

  %  \I What are the consequences of having a real potential for the scattering
  %  wave function and the K-matrix? [\emph{This may be too obscure.}]

   \ee 


 
 \I Exploring the Lippmann-Schwinger equation. [The conventions here follow Taylor.] 
   \be
    
    \I Using the Schr\"odinger equation for the scattering of two particles with mass $m$,
     \beq
       (H_0 + V)|\psi_E\rangle = E |\psi_E\rangle \;,
     \eeq
     where $H_0$ is the free Hamiltonian, show that the Lippmann-Schwinger equation for the wave function,
     \beq
       |\psi_E^{\pm}\rangle = |\phi_k\rangle + \frac{1}{E-H_0\pm i\epsilon}V
         |\psi_E^{\pm}\rangle \;, 
     \eeq
     is satisfied.
     Here $E = k^2/m$ and the plane wave state satisfies $H_0 |\phi_k\rangle = E |\phi_k\rangle$.
     Why do you need the $\pm i\epsilon$?

    
    \I
    We can define the $T$-matrix on-shell as the transition matrix that acting on the plane wave state 
    yields the same result as the potential acting on the full scattering state.  That is,
     $T^{(\pm)}(E = k^2/m)|\phi_k\rangle = V |\psi_E^{\pm}\rangle$. 
    What does it mean that the $T$-matrix is ``on-shell''? (This is a really quick question!) 

    \I
     Show that matrix elements of the $T$-matrix satisfy the Lippmann-Schwinger equation
    \beq
       \langle {\bf k}'|T^{(\pm)}(E)|{\bf k}\rangle =
       \langle {\bf k}'|V|{\bf k}\rangle +
       \int\! d^3p\, \frac{\langle {\bf k}'|V|{\bf p}\rangle
         \langle {\bf p}|T^{(\pm)}(E)|{\bf k}\rangle}{E-\frac{p^2}{m}\pm i\epsilon}
         \;.
    \eeq
    What normalization is used for the momentum states?
    [See the Morrison and A.N. Feldt pedagogical article under Program$\rightarrow$References
    on the webpage.]
    Are the matrix elements of the $T$-matrix on the right side on-shell?

    \I 
    Write the Lippmann-Schwinger equation for the wave function in coordinate space for a local potential $V = V({\bf r})$.
    To this end, show first that the free Green's function 
    \beq
      G^{\pm}({\bf r}',{\bf r}; E = k^2/m) = \langle {\bf r} | \frac{1}{E-H_0\pm i\epsilon} | {\bf r}'\rangle
    \eeq
      is given by 
     \beq
       G^{\pm}({\bf r}',{\bf r}; E = k^2/m) = 
          -\frac{m}{4\pi}\frac{e^{\pm ik|{\bf r}-{\bf r}'|}}{|{\bf r}-{\bf r}'|}
          \;.
     \eeq

    \I Show that when the $T$-matrix is evaluated on-shell, it is proportional to the scattering amplitude, $T^+(E =k^2/m) = -\frac{1}{4\pi^2 m}f(k,\theta)$, by analyzing the asymptotic form of the Lippmann-Schwinger equation and comparing to
    \beq
       \langle {\bf r} | \psi_E^+ \rangle \stackrel{r\rightarrow\infty}{\longrightarrow} 
         (2\pi)^{-3/2} \left( e^{i\bm{k\cdot r}} + f(k,\theta) \frac{e^{ikr}}{r} \right)
       \;.
    \eeq


    \I Start from the momentum-space partial wave expansion of the potential,
    \beq
    \langle {\bf k'} | V | {\bf k} \rangle
    = \frac{2}{\pi}\sum_{l,m} V_l(k',k)Y^{\ast}_{lm}(\Omega_{k'})Y_{lm}(\Omega_k)
    \eeq
    and a similar expansion of the $T$-matrix to 
    derive the partial wave version of the Lippmann-Schwinger equation (with the
    correct factor for the integral):
    \beq
     T_l(k',k;E) = V_l(k',k) + \frac{2}{\pi} \int_0^\infty \! dp\, p^2
      \frac{V_l(k',q)T_l(q,k;E)}{E - p^2/m + i\epsilon} \;.
     \eeq


   \ee


  \I Consider two momentum-space potentials, 
    $V_1(\kvec,\kvec') = V_0\, e^{-(k^2 + k'{}^2)/\mu^2}$ and
    $V_2(\kvec,\kvec') = V_0\, e^{-(\kvec - \kvec')^2/\mu^2}$.
    \be
     \I Are they local or non-local?  
     \I Do they have P-wave projections? (That is, if you wrote it in the partial-wave expansion
     would there be an $L=1$ term?)
     \I Do they have higher angular momentum projections?
    \ee 

  \I Show directly from the Fourier transform expression of a local potential, without
  specifying its functional form, that the momentum space version will only depend on the
 momentum transfer $\kvec'-\kvec$.

  
  \I Scattering phase shifts for a square well
  potential.
   \be
    \I Calculate the S-wave scattering phase shifts for an attractive
    square-well potential $V(r) = -V_0 \theta(R-r)$ and show that
     \beq
       \delta_0(E) = \arctan\left[
         \sqrt{\frac{E}{E+V_0}}\tan\bigl(R\sqrt{2\mu(E+V_0)\bigr)}
         \right] - R\sqrt{2\mu E}
     \eeq

    \I Let's consider the analytic structure of the corresponding partial-wave S matrix,
    which is given by
    \beq
      S_0(k) = e^{-2 i k R} \frac{k_0 \cot k_0 R + ik}{k_0 \cot k_0 R - ik} \,
    \eeq 
    where $E = k^2/2\mu$ and $k_0^2 = k^2 + 2\mu V_0$.
    \be
     \I Show that $S_l(k) = e^{2i\delta_l(k)}$ for $l=0$ is satisfied. 
     [Hint: write $e^{2i\delta} = e^{i\delta}/e^{-i\delta}$.]
     \I Treat $S_0(k)$ as a function of the complex variable $k$ and find its
     singularities.  
     \I
     Bound states are associated with poles on the imaginary axis
     in the upper half plane.  Show that the condition for such a pole here gives
     the same eigenvalue condition (a transcendental equation) that you would get
     from a conventional solution to the square well by matching logarithmic derivatives. 
     [Define $k= i\kappa$ with $\kappa>0$ when analyzing such a pole.]
    \ee
   \ee  

  \I Variable phase approach (VPA) for finding phase shifts from a local
  potential.  Here we consider s-waves.
  [References: Taylor, \textit{Scattering Theory}, pp.~197-201,
  Calogero, \textit{The Variable Phase Approach to Potential Scattering},
  (Academic Press, New York, 1967).]  
   \be
     \I Define the truncated potential $V_\rho(r)$ by
     \beq
        V_\rho(r) = V(r) \theta(\rho-r) \;.
     \eeq
     That is, it is the usual potential for $r \leq \rho$, but identically
     zero beyond that.  Then we define $\delta(k,\rho)$ as the phase shift
     for $V_\rho$ at momentum $k$.  The phase shift we want is
     $\delta(k) = \lim_{\rho\rightarrow\infty} \delta(k,\rho)$.  The basis
     of the variable phase method is a differential equation for $\delta(k,r)$
     at fixed $k$ (again, this is the s-wave equation):
     \beq
       \frac{d\delta(k,r)}{dr} = -\frac{1}{k} 2M V(r) \sin^2[kr + \delta(k,r)]
       \;,
     \eeq
     which is a nonlinear first-order differential equation with initial
     condition $\delta(k,0) = 0$.  Think about how you would implement
     this in your favorite programming language.  
    \I The Mathematica notebook \texttt{square\_well\_scattering.nb} implements
     the VPA for a square well.  Changing to a different potential is
     trivial (see the illustration at the end with a combined short-range repulsive
     square well and a mid-range attractive square well).
     Show that it reproduces the known phase shifts for the square well
     result.
      \I Show from the VPA differential equation that a fully attractive potential gives
     a positive phase shift and a fully negative potential gives a
     negative phase shift. This is the cleanest way to see why the s-wave phaseshifts (which change from positive to negative values at $E_{lab}\approx 270$ MeV in the $^1$S$_0$ partial wave) imply a strong short-range repulsion for local NN potentials.
      \I The VPA automatically builds in Levinson's theorem about the number
     of bound states and the phase shift at zero.  How?  [Hint: what is the
     condition imposed on the phase shift at large energy for Levinson's theorem?  Consider
     integrating $d\delta(k,r)/dr$ in $r$ from zero to infinity.  Use $\sin^2 x \leq 1$
     to put a bound on $\delta(k)$.]
     \I Things to try numerically with the Mathematica or Python notebooks:
       \bi
         \I Try out Levinson's theorem in practice (e.g., for a square well where
         the number of bound states versus depth is easily found in parallel).
         \I Explore the effective range expansion by extracting the $a$ and $r_0$ parameters for 2 different functional forms of $V(r)$ (e.g., square well and a gaussian). Then, try to tune one of the potentials so it gives the same ERE parameters as the other one. 
       \ei

   \ee


 

  \I More on the Lippmann-Schwinger (LS) equation.  
    \be
     \I In the ``Exploring the LS equation'' 
     problem we used the momentum space matrix elements of the operator LS
     equation (we omit the hats here):
       \beq
         T^{(\pm)}(E) = V + V\frac{1}{E - H_0 \pm i\epsilon} T^{(\pm)}(E) \;.
       \eeq

    Show that this can also be written as 
    \beq
               T^{(\pm)}(E) = V + V\frac{1}{E - H \pm i\epsilon} V \;,
    \eeq
    where now the full Green's function appears (it has $H$ instead of $H_0$).
    Do this by repeating the derivation but now using the alternative LS equation
    for the wave function (show that it works!):
     \beq
       |\psi_E^{\pm}\rangle = |\phi_k\rangle + \frac{1}{E-H\pm i\epsilon}V
         |\phi_k\rangle \;. 
     \eeq

   \I Now use the ``spectral representation'' 
   \beq
     \frac{1}{E-H\pm i\epsilon} = \sum_n \frac{|\psi_n\ra \la \psi_n|}{E - E_n}
       + \int\!d^3p\, \frac{|\psi_p^+\ra \la \psi_p^+|}{E - p^2/m \pm i\epsilon}
       \;,
   \eeq
   which follows by inserting a complete set of bound and scattering eigenstates of $H$,
   to show that as a function of energy $E$, the momentum-space $T$-matrix has simple poles at the
   bound-state energies $E_n$ with separable residues $\la \kvec' | V | \psi_n\ra\la \psi_n | V | \kvec\ra$. 
   
   \ee
 \I $T$-matrix for a separable potential [adapted from Taylor, Scattering Theory].  
  A separable potential has the form 
  \beq
             \Vhat = g |\eta\ra \la\eta| \;,
  \eeq
  where we usually choose $|\eta\ra$ to be a normalized vector given, for
  example, by its momentum space function $ \eta(\kvec) \equiv \la\kvec|\eta\ra$
  (note that we're not in partial waves here).
  Recall the Lippmann-Schwinger equation for the operator $T(z)$ :
  \beq
    \That(z) = \Vhat + \Vhat\frac{1}{z-\Hhat_0}\That(z)
      = \Vhat + \Vhat\frac{1}{z-\Hhat_0}\Vhat + 
       \Vhat\frac{1}{z-\Hhat_0}\Vhat\frac{1}{z-\Hhat_0}\Vhat + \cdots 
       \;.
       \label{eq:LS_op}
  \eeq
  \be
    \I Show that $T(z)$ is given explicitly by
      \beq
         T(z) = \frac{g\, |\eta\ra\la\eta|}{1 - g\Delta(z)} \;,
      \eeq
      where
      \beq
        \Delta(z) = \la\eta|\frac{1}{z-H_0}|\eta\ra
          = \int\! d^3k\, \frac{|\eta(\kvec)|^2}{z-E_k} 
      \eeq
      with $E_k = k^2/2\mu$. [Hint: substitute the separable form for $\Vhat$
      into the Born series for $T(z)$ and note the form of each term.]
    \I Show that the Born series for $T(z)$ is convergent for $g$ small
    but divergent for $g$ large.
    \I The poles of $T(z)$ as a function of complex energy $z$ tells of about
    the bound states of the potential.  Show that the separable potential has
    either one or no bound states.  
  \ee

\ee

\end{document}
