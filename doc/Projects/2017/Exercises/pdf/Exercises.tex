%%
%% Automatically generated file from DocOnce source
%% (https://github.com/hplgit/doconce/)
%%
%%


%-------------------- begin preamble ----------------------

\documentclass[%
oneside,                 % oneside: electronic viewing, twoside: printing
final,                   % draft: marks overfull hboxes, figures with paths
10pt]{article}

\listfiles               %  print all files needed to compile this document

\usepackage{relsize,makeidx,color,setspace,amsmath,amsfonts,amssymb}
\usepackage[table]{xcolor}
\usepackage{bm,ltablex,microtype}

\usepackage[pdftex]{graphicx}

\usepackage[T1]{fontenc}
%\usepackage[latin1]{inputenc}
\usepackage{ucs}
\usepackage[utf8x]{inputenc}

\usepackage{lmodern}         % Latin Modern fonts derived from Computer Modern

% Hyperlinks in PDF:
\definecolor{linkcolor}{rgb}{0,0,0.4}
\usepackage{hyperref}
\hypersetup{
    breaklinks=true,
    colorlinks=true,
    linkcolor=linkcolor,
    urlcolor=linkcolor,
    citecolor=black,
    filecolor=black,
    %filecolor=blue,
    pdfmenubar=true,
    pdftoolbar=true,
    bookmarksdepth=3   % Uncomment (and tweak) for PDF bookmarks with more levels than the TOC
    }
%\hyperbaseurl{}   % hyperlinks are relative to this root

\setcounter{tocdepth}{2}  % levels in table of contents

% --- fancyhdr package for fancy headers ---
\usepackage{fancyhdr}
\fancyhf{} % sets both header and footer to nothing
\renewcommand{\headrulewidth}{0pt}
\fancyfoot[LE,RO]{\thepage}
% Ensure copyright on titlepage (article style) and chapter pages (book style)
\fancypagestyle{plain}{
  \fancyhf{}
  \fancyfoot[C]{{\footnotesize \copyright\ 1999-2017, Nuclear Forces PHY989. Released under CC Attribution-NonCommercial 4.0 license}}
%  \renewcommand{\footrulewidth}{0mm}
  \renewcommand{\headrulewidth}{0mm}
}
% Ensure copyright on titlepages with \thispagestyle{empty}
\fancypagestyle{empty}{
  \fancyhf{}
  \fancyfoot[C]{{\footnotesize \copyright\ 1999-2017, Nuclear Forces PHY989. Released under CC Attribution-NonCommercial 4.0 license}}
  \renewcommand{\footrulewidth}{0mm}
  \renewcommand{\headrulewidth}{0mm}
}

\pagestyle{fancy}


% prevent orhpans and widows
\clubpenalty = 10000
\widowpenalty = 10000

\newenvironment{doconceexercise}{}{}
\newcounter{doconceexercisecounter}


% ------ header in subexercises ------
%\newcommand{\subex}[1]{\paragraph{#1}}
%\newcommand{\subex}[1]{\par\vspace{1.7mm}\noindent{\bf #1}\ \ }
\makeatletter
% 1.5ex is the spacing above the header, 0.5em the spacing after subex title
\newcommand\subex{\@startsection*{paragraph}{4}{\z@}%
                  {1.5ex\@plus1ex \@minus.2ex}%
                  {-0.5em}%
                  {\normalfont\normalsize\bfseries}}
\makeatother


% --- end of standard preamble for documents ---


% insert custom LaTeX commands...

\raggedbottom
\makeindex
\usepackage[totoc]{idxlayout}   % for index in the toc
\usepackage[nottoc]{tocbibind}  % for references/bibliography in the toc

%-------------------- end preamble ----------------------

\begin{document}

% matching end for #ifdef PREAMBLE

\newcommand{\exercisesection}[1]{\subsection*{#1}}


% ------------------- main content ----------------------



% ----------------- title -------------------------

\thispagestyle{empty}

\begin{center}
{\LARGE\bf
\begin{spacing}{1.25}
Selected Exercises
\end{spacing}
}
\end{center}

% ----------------- author(s) -------------------------

\begin{center}
{\bf Nuclear Forces PHY989}
\end{center}

    \begin{center}
% List of all institutions:
\centerline{{\small National Superconducting Cyclotron Laboratory and Department of Physics and Astronomy, Michigan State University, East Lansing, USA}}
\end{center}
    
% ----------------- end author(s) -------------------------

% --- begin date ---
\begin{center}
Fall semester 2017
\end{center}
% --- end date ---

\vspace{1cm}


% --- begin exercise ---
\begin{doconceexercise}
\refstepcounter{doconceexercisecounter}

\exercisesection*{Exercise \thedoconceexercisecounter: Questions on the Overview of QCD}


This collection of problems contain short exercises and discussion questions on QCD.


\subex{a)}
With respect to what scale(s) are the $c,b,t$ quarks called heavy?

\subex{b)}
Have you heard about the $s$ quark before? If yes, in what context?

\subex{c)}
A possible way to \emph{see} quarks and gluons is in jets. What happens in these events?

\subex{d)}
Using the \href{{http://pdg.lbl.gov/}}{Particle Data Group website}, discuss which properties of the
neutron and proton are similar and what are differences? What about for the three pions?

\subex{e)}
Which is more important in making a neutron more massive than a proton: the light quark mass difference or the electromagnetic
contribution? Or do you think such considerations are too simplistic?

\subex{f)}
What is the evidence for \emph{spontaneous} chiral symmetry  breaking in
\begin{itemize}
   \item the mass spectrum of pseudoscalar ($J^\pi = 0^-$) mesons;

   \item the mass spectrum of vector and axial vector ($J^\pi = 1^\mp$) mesons?
\end{itemize}

\noindent
\subex{g)}
What is the evidence for \emph{explicit} chiral symmetry breaking
      in the  spectrum of pseudoscalar ($J^\pi = 0^-$) mesons?

\subex{h)}
If you and your friend each do a QCD calculation with the
      same diagrams but use $\alpha_s$ at different scales, will you
      get the same answer? If not, how could that happen?

\subex{i)}
Does the running coupling in QCD mean that the QCD
      Hamiltonian is not unique? Would you say that if you used
      $\alpha_s$ at two different scales that you were using two
      different Hamiltonians?

\subex{j)}
If the neutron lifetime is so short, why are there \emph{any} stable nuclei?

\subex{k)}
One observes a marked resonance when a $\pi^+$ pion is
      scattering off a proton. Which baryon does this correspond to
      and at which energy of the $\pi^+$ does this occur (the proton
      is at rest)?

\subex{l)}
At sufficient energy in proton-proton collisions it is
      possible to create a pion, $p + p \rightarrow p + n + \pi^+$. At
      which energy in the center-of-mass frame does pion production
      start?

\end{doconceexercise}
% --- end exercise ---




% --- begin exercise ---
\begin{doconceexercise}
\refstepcounter{doconceexercisecounter}

\exercisesection*{Exercise \thedoconceexercisecounter: Basic Scattering Theory}



\subex{a)}
We typically use units in which $\hbar=c=1$ and express
      quantities as powers of MeV or fm or both, using $\hbar c\approx 197.33$ MeVfm to convert between them. If we take for the nucleon
      mass $M_N=939$ MeV/$c^2$, what is $\hbar^2/M_N$ numerically
      in terms of MeV and fm?

% --- begin hint in exercise ---

\paragraph{Hint.}
Hint: This should be almost immediate if you insert the right factors of $c$.

% --- end hint in exercise ---

\subex{b)}
For the scattering of equal mass (nonrelativistic) particles, 
      if the laboratory energy $E_{\rm lab}$ is related to the magnitude of the relative momentum 
      $k_{\rm rel}$ (i.e., the momentum each particle has in the center-of-mass
      frame) by $E_{\rm lab} = C k_{\rm rel}^2$, what is $C$?  If the mass
      is $M_N=939$ MeV, what is the value of $C$ in MeVfm$^2$?

\subex{c)}
We write the partial-wave momentum space Schroedinger equation (\href{{https://manybodyphysics.github.io/NuclearForces/doc/pub/scatteringtheory/html/scatteringtheory.html}}{see Lecture notes}) as
\[
\frac{k^2}{2\mu}\langle klm | \psi \rangle + \frac{2}{\pi}\sum_{l'm'}
        \int_0^\infty\! dk'\,k'{}^2\, 
        \langle klm | V | k'l'm' \rangle \langle k'l'm' | \psi \rangle = E_k \langle klm | \psi \rangle \;,
\]
      what are the units of $V_{ll'}(k,k') \equiv   \langle klm | V | k'l'm \rangle$? In coordinate space the potential is local, $V(r)$, with units of MeV, and $k$ is given in inverse fm.  If you see a plot in a journal
      article
      of $V_{ll'}(k,k')$ with units of fm, how would you convert it to the units you just found?

% --- begin hint in exercise ---

\paragraph{Hint.}
Hint: use the results from the first exercise here.

% --- end hint in exercise ---

\subex{d)}
In Figure 18 of the review by Scott Bogner \emph{et al.},
      \href{{http://www.sciencedirect.com/science/article/pii/S0146641010000347?via%3Dihub}}{Prog.  Nucl. Part. Phys. \textbf{65}, 94 (2010)} the momentum-space
      matrix elements of different chiral effective field theory
      potentials are given in units of fm. Consider the value at zero
      relative momenta. 
      $\tilde{C}_{^1S_0}$, see Eq.~(2.5) and 
     the \href{{http://www.sciencedirect.com/science/article/pii/S0375947404010747}}{article by Epelbaum} \emph{et al.}  in GeV$^{-2}$. How do you convert
      to fm units? Do the values for the matrix elements then match?

\subex{e)}
What do \emph{on-shell} and \emph{off-shell} mean in the context of scattering?

\subex{f)}
Under what conditions is a partial-wave expansion of the potential useful?

\subex{g)}
Derive the standard result:
\[
       \frac{e^{i\delta_l(k)}\sin\delta_l(k)}{k}
            = \frac{1}{k\cot\delta_l(k) - i k}
\]

\subex{h)}
Given a potential that is not identically zero as $r\rightarrow\infty$ (e.g., a Yukawa),
     how would you know in practice where the asymptotic (large $r$) region starts?

\subex{i)}
What is the physical interpretation of the relation between the (partial-wave)
    $S$-matrix and the scattering amplitude?  (Note that $S_l(k) = 1 + 2 i k f_l(k)$.)


\end{doconceexercise}
% --- end exercise ---




% --- begin exercise ---
\begin{doconceexercise}
\refstepcounter{doconceexercisecounter}

\exercisesection*{Exercise \thedoconceexercisecounter: More on the Lippmann-Schwinger equation}



\subex{a)}
Using the Schr\"odinger equation for the scattering of two particles with mass $m$,
\[
       (H_0 + V)|\psi_E\rangle = E |\psi_E\rangle \;,
\]
     where $H_0$ is the free Hamiltonian, show that the Lippmann-Schwinger equation for the wave function,
\[
       |\psi_E^{\pm}\rangle = |\phi_k\rangle + \frac{1}{E-H_0\pm i\epsilon}V
         |\psi_E^{\pm}\rangle \;, 
\]
     is satisfied.
     Here $E = k^2/m$ and the plane wave state satisfies $H_0 |\phi_k\rangle = E |\phi_k\rangle$.
     Why do you need the $\pm i\epsilon$?

\subex{b)}
We can define the $T$-matrix on-shell as the transition matrix that acting on the plane wave state 
    yields the same result as the potential acting on the full scattering state.  That is,
     $T^{(\pm)}(E = k^2/m)|\phi_k\rangle = V |\psi_E^{\pm}\rangle$. 
    What does it mean that the $T$-matrix is \emph{on-shell}? (This is a really quick question!)

\subex{c)}
Show that matrix elements of the $T$-matrix satisfy the Lippmann-Schwinger equation
\[
       \langle {\bf k}'|T^{(\pm)}(E)|{\bf k}\rangle =
       \langle {\bf k}'|V|{\bf k}\rangle +
       \int\! d^3p\, \frac{\langle {\bf k}'|V|{\bf p}\rangle
         \langle {\bf p}|T^{(\pm)}(E)|{\bf k}\rangle}{E-\frac{p^2}{m}\pm i\epsilon}.
\]
    What normalization is used for the momentum states?
    Are the matrix elements of the $T$-matrix on the right side on-shell?

\subex{d)}
Write the Lippmann-Schwinger equation for the wave function in coordinate space for a local potential $V = V({\bf r})$.
    To this end, show first that the free Green's function 
\[
      G^{\pm}({\bf r}',{\bf r}; E = k^2/m) = \langle {\bf r} | \frac{1}{E-H_0\pm i\epsilon} | {\bf r}'\rangle,
\]
is given by 
\[
       G^{\pm}({\bf r}',{\bf r}; E = k^2/m) = 
          -\frac{m}{4\pi}\frac{e^{\pm ik|{\bf r}-{\bf r}'|}}{|{\bf r}-{\bf r}'|}.
\]

\subex{e)}
Show that when the $T$-matrix is evaluated on-shell, it is proportional to the scattering amplitude, $T^+(E =k^2/m) = -\frac{1}{4\pi^2 m}f(k,\theta)$, by analyzing the asymptotic form of the Lippmann-Schwinger equation and comparing to
\[
       \langle {\bf r} | \psi_E^+ \rangle \stackrel{r\rightarrow\infty}{\longrightarrow} 
         (2\pi)^{-3/2} \left( e^{i\bm{k\cdot r}} + f(k,\theta) \frac{e^{ikr}}{r} \right).
 \]

\subex{f)}
Start from the momentum-space partial wave expansion of the potential,
\[
    \langle {\bf k'} | V | {\bf k} \rangle
    = \frac{2}{\pi}\sum_{l,m} V_l(k',k)Y^{\ast}_{lm}(\Omega_{k'})Y_{lm}(\Omega_k),
\]
    and a similar expansion of the $T$-matrix to 
    derive the partial wave version of the Lippmann-Schwinger equation (with the
    correct factor for the integral):
\[
     T_l(k',k;E) = V_l(k',k) + \frac{2}{\pi} \int_0^\infty \! dp\, p^2
      \frac{V_l(k',q)T_l(q,k;E)}{E - p^2/m + i\epsilon}.
\]


\subex{g)}
Scattering phase shifts for a square well potential. Calculate the S-wave scattering phase shifts for an attractive
    square-well potential $V(r) = -V_0 \theta(R-r)$ and show that
\[
       \delta_0(E) = \arctan\left[
         \sqrt{\frac{E}{E+V_0}}\tan\bigl(R\sqrt{2\mu(E+V_0)\bigr)}
         \right] - R\sqrt{2\mu E}.
\]


\subex{h)}
Let's consider the analytic structure of the corresponding partial-wave S matrix,
    which is given by
\[
      S_0(k) = e^{-2 i k R} \frac{k_0 \cot k_0 R + ik}{k_0 \cot k_0 R - ik},
\]
where $E = k^2/2\mu$ and $k_0^2 = k^2 + 2\mu V_0$.
Show that $S_l(k) = e^{2i\delta_l(k)}$ for $l=0$ is satisfied. 
Treat $S_0(k)$ as a function of the complex variable $k$ and find its  singularities.

% --- begin hint in exercise ---

\paragraph{Hint.}
Hint: write $e^{2i\delta} = e^{i\delta}/e^{-i\delta}$.

% --- end hint in exercise ---

\subex{i)}
Bound states are associated with poles on the imaginary axis
     in the upper half plane.  Show that the condition for such a pole here gives
     the same eigenvalue condition (a transcendental equation) that you would get
     from a conventional solution to the square well by matching logarithmic derivatives.

% --- begin hint in exercise ---

\paragraph{Hint.}
Hint: Define $k= i\kappa$ with $\kappa>0$ when analyzing such a pole.

% --- end hint in exercise ---

\end{doconceexercise}
% --- end exercise ---


% ------------------- end of main content ---------------

\end{document}

