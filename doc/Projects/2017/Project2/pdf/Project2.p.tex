%%
%% Automatically generated file from DocOnce source
%% (https://github.com/hplgit/doconce/)
%%
%%
% #ifdef PTEX2TEX_EXPLANATION
%%
%% The file follows the ptex2tex extended LaTeX format, see
%% ptex2tex: http://code.google.com/p/ptex2tex/
%%
%% Run
%%      ptex2tex myfile
%% or
%%      doconce ptex2tex myfile
%%
%% to turn myfile.p.tex into an ordinary LaTeX file myfile.tex.
%% (The ptex2tex program: http://code.google.com/p/ptex2tex)
%% Many preprocess options can be added to ptex2tex or doconce ptex2tex
%%
%%      ptex2tex -DMINTED myfile
%%      doconce ptex2tex myfile envir=minted
%%
%% ptex2tex will typeset code environments according to a global or local
%% .ptex2tex.cfg configure file. doconce ptex2tex will typeset code
%% according to options on the command line (just type doconce ptex2tex to
%% see examples). If doconce ptex2tex has envir=minted, it enables the
%% minted style without needing -DMINTED.
% #endif

% #define PREAMBLE

% #ifdef PREAMBLE
%-------------------- begin preamble ----------------------

\documentclass[%
oneside,                 % oneside: electronic viewing, twoside: printing
final,                   % draft: marks overfull hboxes, figures with paths
10pt]{article}

\listfiles               %  print all files needed to compile this document

\usepackage{relsize,makeidx,color,setspace,amsmath,amsfonts,amssymb}
\usepackage[table]{xcolor}
\usepackage{bm,ltablex,microtype}

\usepackage[pdftex]{graphicx}

\usepackage[T1]{fontenc}
%\usepackage[latin1]{inputenc}
\usepackage{ucs}
\usepackage[utf8x]{inputenc}

\usepackage{lmodern}         % Latin Modern fonts derived from Computer Modern

% Hyperlinks in PDF:
\definecolor{linkcolor}{rgb}{0,0,0.4}
\usepackage{hyperref}
\hypersetup{
    breaklinks=true,
    colorlinks=true,
    linkcolor=linkcolor,
    urlcolor=linkcolor,
    citecolor=black,
    filecolor=black,
    %filecolor=blue,
    pdfmenubar=true,
    pdftoolbar=true,
    bookmarksdepth=3   % Uncomment (and tweak) for PDF bookmarks with more levels than the TOC
    }
%\hyperbaseurl{}   % hyperlinks are relative to this root

\setcounter{tocdepth}{2}  % levels in table of contents

% --- fancyhdr package for fancy headers ---
\usepackage{fancyhdr}
\fancyhf{} % sets both header and footer to nothing
\renewcommand{\headrulewidth}{0pt}
\fancyfoot[LE,RO]{\thepage}
% Ensure copyright on titlepage (article style) and chapter pages (book style)
\fancypagestyle{plain}{
  \fancyhf{}
  \fancyfoot[C]{{\footnotesize \copyright\ 1999-2017, Nuclear Forces PHY989. Released under CC Attribution-NonCommercial 4.0 license}}
%  \renewcommand{\footrulewidth}{0mm}
  \renewcommand{\headrulewidth}{0mm}
}
% Ensure copyright on titlepages with \thispagestyle{empty}
\fancypagestyle{empty}{
  \fancyhf{}
  \fancyfoot[C]{{\footnotesize \copyright\ 1999-2017, Nuclear Forces PHY989. Released under CC Attribution-NonCommercial 4.0 license}}
  \renewcommand{\footrulewidth}{0mm}
  \renewcommand{\headrulewidth}{0mm}
}

\pagestyle{fancy}


% prevent orhpans and widows
\clubpenalty = 10000
\widowpenalty = 10000

% --- end of standard preamble for documents ---


% insert custom LaTeX commands...

\raggedbottom
\makeindex
\usepackage[totoc]{idxlayout}   % for index in the toc
\usepackage[nottoc]{tocbibind}  % for references/bibliography in the toc

%-------------------- end preamble ----------------------

\begin{document}

% matching end for #ifdef PREAMBLE
% #endif

\newcommand{\exercisesection}[1]{\subsection*{#1}}


% ------------------- main content ----------------------



% ----------------- title -------------------------

\thispagestyle{empty}

\begin{center}
{\LARGE\bf
\begin{spacing}{1.25}
Project 2, deadline  December 15
\end{spacing}
}
\end{center}

% ----------------- author(s) -------------------------

\begin{center}
{\bf Nuclear Forces PHY989}
\end{center}

    \begin{center}
% List of all institutions:
\centerline{{\small National Superconducting Cyclotron Laboratory and Department of Physics and Astronomy, Michigan State University, East Lansing, USA}}
\end{center}
    
% ----------------- end author(s) -------------------------

% --- begin date ---
\begin{center}
Fall semester 2017
\end{center}
% --- end date ---

\vspace{1cm}


\subsection{Effective field theory}

The aim of this project is to build your own Effective field theory
(EFT) interaction model that approximates the phase shifts you
extracted with the simple interaction model from project 1. We will
thus use as input the phase shift analysis from the previous project
and use these as our theoretical benchmark data.  We will follow closely
\href{{https://arxiv.org/abs/nucl-th/9706029}}{Lepage's article *How to renormalize the Schroedinger
equation*}, see especially
pages 8-17.


We will start by building a \emph{pionless} EFT potential to describe
nucleon-nucleon  scattering low energies where even the longest range piece of the
potential is unresolved.  Therefore, the EFT potential is just a
series of contact interactions. Recall that this is the domain of the
effective range expansion, which is valid up to an onshell momentum
$k\sim m_{\pi}/2$, which  translates to about 10 MeV lab energy. If we
were considering the \emph{true} NN system with non-central terms (and full
spin- and isospin-dependent terms, we drop isospin here), then the contact interactions take
the form: 
\begin{align*}
V^\mathrm{LO}(\vec{q},\vec{k})=&C_S+C_T\vec{\sigma}_1\cdot\vec{\sigma}_2,\\
V^\mathrm{NLO}(\vec{q},\vec{k})=&C_1\vec{q}^{\,2}+C_2\vec{k}^2+
\vec{\sigma}_1\cdot\vec{\sigma}_2\left(C_3\vec{q}^2+C_4\vec{k}^2\right)\\
&+iC_5\frac{\vec{\sigma}_1+\vec{\sigma}_2}{2}\cdot\vec{q}\times\vec{k}+C_6\vec{q}\cdot\vec{\sigma}_1\vec{q}\cdot\vec{\sigma}_2
+C_7\vec{k}\cdot\vec{\sigma}_1\vec{k}\cdot\vec{\sigma}_2, 
\end{align*}
where $\vec{q}=\vec{p}-\vec{p'}$ is the momentum transfer,
$\vec{k}=\frac{\vec{p}+\vec{p'}}{2}$ the average momentum, and
$\vec{p},\vec{p'}$ the relative momenta.  However, since our
``underlying theory'' in Project 1 is spin- and isospin-independent,
the above simplifies substantially, just strike out any terms with
Pauli spin/isospin matrices!




\paragraph{Project 2a): Lowest-order pionless theory.}
We start by fitting the lowest order pionless interaction model. Projecting into the $^1S_0$ channel, the potential is given by (excluding spin/isospin-dependent terms)
\[
V^\mathrm{LO}_{^1S_0}(p',p)=C_0.
\]

Note that the coupling constants appearing in the $^1S_0$
expressions are linear combinations of the coupling constants in the
general expressions shown above. Their exact form will not concern us here,
though you can find the relevant details on pp. 32-33 of \href{{https://arxiv.org/abs/1105.2919}}{Machleidt and Entem}.

As it stands, this potential would give UV
divergences if you naively used it in the LS equation of project 1. Therefore, we
need to regularize the potential to cutoff the problematic
high-momentum modes in the loop integrals. In class, we used a sharp
cutoff so that the above would by multiplied by $\theta(\Lambda -
p)\theta(\Lambda-p')$. While conceptually simple, sharp cutoffs can be
tricky, can you guess how you would need to modify the \emph{trick} used
to handle the principal value integral in the LS equation? (This is
actually the least of our worries. If we try to use a sharp cutoff in
some manybody method that relies on expanding the Hamiltonian in a
harmonic oscillator basis, the expansion is very slowly converging in
analogy with the \emph{Gibbs overshoot} phenomena in Fourier analysis.) To
avoid these subtleties, we will instead use a smooth UV regulator as

\[
V(p',p)\Rightarrow f_{\Lambda}(p')V(p',p)f_{\Lambda}(p)\,
\] 
where we choose
\[
f_{\Lambda}(p) = \exp{[-p^4/\Lambda^4]}\,.
\]

Choose an \emph{appropriate} value for $\Lambda$ and fit $C_0$ to low
energy phase shifts. Hint: Following Lepage, always use the \emph{most infrared} data possible when fitting the couplings of contact
terms. Also, remember Lepage's discussion on choosing the value for
$\Lambda$.


\paragraph{Project 2b): Still pionless, but adding the next order.}
We add now the NLO correction term
\[
V^\mathrm{NLO}_{^1S_0}(p',p)=C_2(p^2 + p'^2),
\]
and fit the parameters ($C_0$ and $C_2$) to the low-energy phase shifts from project 1. Next, we add the NNLO correction term
\[
V^\mathrm{NNLO}_{^1S_0}(p',p)=C_4(p^4 + p'^4) + C'_4 p^2p'^2,
\]

and fit the four parameters $C_0,C_2,C_4,C'_4$ to the low-energy phase
shifts.  Make a plot of your calculated LO, NLO, and NNLO phase shifts
and compare them to the results from the Project 1 potential. Also
make a plot a-la Fig.~3 in Lepage to show the power-law improvement
with each additional order, and from this pick off the breakdown scale
$\Lambda_b$ where the EFT stops improving with additional
orders. These log-log error plots are commonly known as \emph{Lepage plots}, and they help show that an EFT is working as expected.

\paragraph{Project 2c): Cutoff dependence.}
Repeat the NLO pionless fits for several different values of $\Lambda$, and make the analogous plot to Fig.~4 in Lepage. If we denote a generic momentum of some low-energy process by $Q$, then the theoretical errors in an EFT calculation scale roughly as 
\[
\Delta(Q) \sim {\rm Max}\,{((Q/\Lambda)^n, (Q/\Lambda_b)^n)},
\]
which is why taking $\Lambda >> \Lambda_b$ doesn't really buy you anything. 

\paragraph{Project 2d): Including the \emph{one-pion exchange} term.}
The pionless EFT basically reproduces the physics of the Effective
Range Expansion, so we expect $\Lambda_b\sim m_{\pi}$, which sets the
scale of the longest-ranged Yukawa term in the toy potential of
Project 1. We now want to build a toy EFT which works at higher
energies. To do this, we now include the longest-ranged Yukawa
potential as an explicit degree of freedom in our EFT. (We should put
quotation marks around the one-pion exchange moniker since our
underlying theory is a toy model, without the full spin/isospin
dependence that would come with the true OPE potential.)

We will repeat parts 2a)-2b) by including the \emph{simple one-pion exchange}
contribution and refitting the contact term couplings. Therefore, at
LO our EFT potential takes the form
\[
V^\mathrm{LO}(p,p')= V_{\pi}(p,p') + C_0\,
\]
while at NLO our potential takes the form
\[
V^\mathrm{NLO}(p,p')= V_{\pi}(p,p') + C_0 + C_2(p^2+p'^2)\,,
\]
with the analogous form for the NNLO potential. 

Redo the analysis of parts 2a-2b, taking care to modify your choice of
$\Lambda$ to an \emph{appropriate} value.  Your analysis should be very
similar as for the pionless theory, except that the value of
$\Lambda_b$ should be at a higher value, roughly equal to the mass of
the next heaviest Yukawa term.

\paragraph{Project 2e (optional)): Cutoff dependence.}
Redo the analysis of part 2c, and comment on any similarities/differences. 


\subsection{Introduction to numerical projects}

Here follows a brief recipe and recommendation on how to write a report for each
project.

\begin{itemize}
  \item Give a short description of the nature of the problem and the eventual  numerical methods you have used.

  \item Describe the algorithm you have used and/or developed. Here you may find it convenient to use pseudocoding. In many cases you can describe the algorithm in the program itself.

  \item Include the source code of your program. Comment your program properly.

  \item If possible, try to find analytic solutions, or known limits in order to test your program when developing the code.

  \item Include your results either in figure form or in a table. Remember to        label your results. All tables and figures should have relevant captions        and labels on the axes.

  \item Try to evaluate the reliabilty and numerical stability/precision of your results. If possible, include a qualitative and/or quantitative discussion of the numerical stability, eventual loss of precision etc.

  \item Try to give an interpretation of you results in your answers to  the problems.

  \item Critique: if possible include your comments and reflections about the  exercise, whether you felt you learnt something, ideas for improvements and  other thoughts you've made when solving the exercise. We wish to keep this course at the interactive level and your comments can help us improve it.

  \item Try to establish a practice where you log your work at the  computerlab. You may find such a logbook very handy at later stages in your work, especially when you don't properly remember  what a previous test version  of your program did. Here you could also record  the time spent on solving the exercise, various algorithms you may have tested or other topics which you feel worthy of mentioning.
\end{itemize}

\noindent
\subsection{Format for electronic delivery of report and programs}

The preferred format for the report is a PDF file. You can also use DOC or postscript formats or as an ipython notebook file.  As programming language we prefer that you choose between C/C++, Fortran2008 or Python. The following prescription should be followed when preparing the report:

\begin{itemize}
  \item Use your github repository to upload your report. Indicate where the report is by creating for example a \textbf{Report} folder. Please send us as soon as possible your github username.

  \item Place your programs in a folder called for example \textbf{Programs} or \textbf{src}, in order to indicate where your programs are. You can use a README file to tell us how your github folders are organized. 

  \item In your git repository, please include a folder which contains selected results. These can be in the form of output from your code for a selected set of runs and input parameters.

  \item In this and all later projects, you should include tests (for example unit tests) of your code(s).

  \item Comments  from us on your projects, with score and detailed feedback will be emailed to you. 
\end{itemize}

\noindent
Finally, 
we encourage you to work two and two together. Optimal working groups consist of 
2-3 students. You can then hand in a common report. 








% ------------------- end of main content ---------------

% #ifdef PREAMBLE
\end{document}
% #endif

